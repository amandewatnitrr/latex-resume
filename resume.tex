\documentclass{article}

\begin{document}
	
	\title{Critique of Suppression of Freedom of Expression by the Government in India}
	\author{Aman Kumar Dewangan}
	\date{\today}
	
	\maketitle
	
	\section{Introduction}
	Freedom of expression is a fundamental right that forms the cornerstone of any democratic society. It allows individuals to voice their opinions, criticize policies, and hold the government accountable. However, recent incidents in India suggest that the government is suppressing this essential right, particularly when it comes to questioning government actions through the Right to Information (RTI) Act and expressing dissent through mass media. This document delves into three distinct cases that illustrate the erosion of freedom of expression and the implications for democracy.
	
	\section{Case 1: IIT Kharagpur's Suppression of Faculty Voices}
	The report from Business Insider reveals that the Indian Institute of Technology (IIT) Kharagpur allegedly instructed its faculty not to write on topics critical of the government. This incident signifies a worrying trend wherein academic institutions, meant to foster intellectual freedom and critical thinking, are seemingly muzzled by the ruling authorities.
	{\bigskip}
	
	The suppression of faculty members' voices undermines academic autonomy and stifles scholarly discourse. By imposing limitations on what academics can write, the government risks hindering the pursuit of knowledge and the objective analysis of policy decisions. This undermines the essence of a vibrant democracy, where informed and critical perspectives are essential for informed decision-making.
	
	\section{Case 2: Ashoka University's Research Paper Controversy}
	The incident involving Ashoka University, as reported by NDTV, highlights the government's efforts to control the narrative and quell dissenting voices even in private institutions. The controversy erupted over a research paper by a faculty member, leading to a political row.
	
	Such incidents cast a shadow on the autonomy of educational institutions, as they are expected to be spaces where diverse opinions are welcomed and nurtured. The government's interference in academic affairs raises questions about the independence of institutions from political pressure, potentially deterring researchers and scholars from tackling critical issues. This undermines the spirit of open inquiry and the role of academia in contributing to societal progress.
	
	\section{Case 3: Unacademy's Firing of Educator for Political Expression}
	The incident involving Unacademy, as reported by Careers360, underscores the delicate balance between personal opinions and professional consequences. Unacademy, a prominent online learning platform, sacked a teacher named Karan Sangwan for urging people to vote for educated leaders and calling for the uninstallation of the Unacademy app.
	
	The episode raises concerns about the freedom of expression of individuals outside their professional realms. While private companies have the right to establish conduct policies, the incident suggests that even expressions of political opinions are not immune from repercussions. This has chilling effects on citizens' ability to engage in political discourse without fear of retribution.
	
	\section{Conclusion}
	These three cases illustrate the multifaceted suppression of freedom of expression by the government in India. Whether through academic institutions, private companies, or independent media, the right to voice opinions, question the government, and express dissent is under threat. Such actions not only undermine the democratic fabric of the nation but also stifle innovation, intellectual growth, and informed decision-making.
	
	In a thriving democracy, a healthy exchange of ideas, even critical ones, is essential for growth and progress. The government's attempts to suppress voices critical of its policies and actions do a disservice to the citizens it is meant to serve. It is crucial to safeguard the right to expression and maintain an environment where diverse opinions can coexist and contribute to a more informed, inclusive, and resilient society.
	
\end{document}
